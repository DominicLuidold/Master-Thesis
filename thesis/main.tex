%% Metadaten
\begin{filecontents}{\jobname.xmpdata}
    \Title{TODO}
    \Author{Dominic Luidold, BSc}
    \Language{de-AT}
    \Keywords{TODO}
\end{filecontents}

\documentclass[a4paper,12pt,twoside]{scrreprt}
% Autor der Vorlage: Klaus Rheinberger, FH Vorarlberg, 2017-02-20

%% Pakete
% Dokumenteneigenschaften
\usepackage[ngerman]{babel}                 % Deutsche Sprachanpassungen
\usepackage[T1]{fontenc}                    % Silbentrennung bei Sonderzeichen
\usepackage[bindingoffset=8mm]{geometry}    % Bindeverlust von 8mm einbeziehen
\usepackage{minted}                         % Code Highlighting/Import
\usepackage[a-3b,mathxmp]{pdfx}[2018/12/22] % PDF/A
\usepackage{setspace}                       % Zeilenabstand

% Bilder
\usepackage{graphicx} % Bilder einbinden
\usepackage{wrapfig}  % Bilder positionieren

% Zitate & Verweise, Sonstiges
\usepackage[nohyperlinks]{acronym}  % Abkürzungsverzeichnis
\usepackage{caption}                % Abbildungslegenden
\usepackage{csquotes}               % Anführungszeichen und Zitieren
\usepackage[
    style=ieee,
    backend=biber
]{biblatex}                         % Literaturverweise
\usepackage{xcolor}                 % Farbige Hervorhebungen

%% Einstellungen
\addbibresource{references.bib}
\onehalfspacing
\setcounter{secnumdepth}{4} % Nummerierungstiefe
\setcounter{tocdepth}{3}    % Gliederungstiefe Inhaltsverzeichnis

%% Dokument
\begin{document}

% Titelblatt
\pagenumbering{roman}

\begin{titlepage}
    \begin{flushright}
    \includegraphics[width=0.4\linewidth]{images/FHV_FHV-Logo.png}
    \end{flushright}

    \begin{flushleft}
    \section*{[Titel der Arbeit]}
    \subsection*{[Untertitel der Arbeit]}
    \vspace{1cm}

    Masterarbeit\\
    zur Erlangung des akademischen Grades
    \vspace{0.5cm}

    \textbf{Master of Science in Engineering (MSc)}

    \vspace{1cm}
    Fachhochschule Vorarlberg\newline
    Informatik Master

    \vspace{0.5cm}

    Betreut von\newline
    Karin Trommelschläger, MSc

    \vspace{0.5cm}

    Vorgelegt von\newline
    Dominic Luidold, BSc\newline
    Dornbirn, [Monat Jahr]
    \end{flushleft}
\end{titlepage}

% Widmung
\newpage
\section*{Widmung}
\label{sec:widmung}

\colorbox{yellow}{TODO - Widmung}

% Abstract [DE]
\newpage
\section*{Kurzreferat}
\label{sec:abstract-de}

\subsection*{Titel [DE]}

\colorbox{yellow}{TODO - Kurzreferat}

% Abstract [EN]
\newpage
\section*{Abstract}
\label{sec:abstract-en}

\subsection*{Titel [EN]}

\colorbox{yellow}{TODO - Abstract}

% Inhaltsverzeichnis
\cleardoublepage
\tableofcontents

% Abbildungsverzeichnis
\clearpage
\phantomsection
\addcontentsline{toc}{chapter}{Abbildungsverzeichnis}
\listoffigures

% Abkürzungsverzeichnis
\clearpage
\phantomsection
\addcontentsline{toc}{chapter}{Abkürzungsverzeichnis}
\chapter*{Abkürzungsverzeichnis}

\begin{acronym}
    \acro{poc}[PoC]{Proof of Concept}
    \acro{ux}[UX]{User Experience}
\end{acronym}

% Inhalt
\cleardoublepage
\pagenumbering{arabic}

\chapter{Einleitung}
\label{chap:einleitung}

Die vorliegende Masterarbeit beschäftigt sich mit dem Erstellen und Einreichen von Ethikanträgen, die bei der Forschungsethik-Kommission der Fachhochschule Vorarlberg\footnote{\href{https://www.fhv.at/forschung/forschungsethik-kommission/}{Forschungsethik-Kommission: Analyse von Forschungstätigkeiten und deren ethischer Grundsatzfragen (https://www.fhv.at/forschung/forschungsethik-kommission/)}} eingereicht werden können. Ziel dieser Arbeit ist es, Lösungsvorschläge auszuarbeiten, die das auf Word-Dokumentenvorlagen aufbauende System langfristig ablösen können.

\medskip

Um ein grundlegendes Verständnis über die Thematik sowie den Ablauf zu erlangen, wird zu Beginn der Arbeit auf den Aufgabenbereich und die Arbeitsweise der Forschungsethik-Kommission eingegangen sowie der aktuelle Prozess rund um das Erstellen und Einreichen von Ethikanträgen beleuchtet. Dabei werden durch gezielt geführte Interviews mit Mitgliedern der Forschungsethik-Kommission sowie potenziellen und ehemaligen Antragssteller:innen allfällige Schwachstellen aufgearbeitet und etwaige Wünsche für ein neues System erörtert. Im Zuge der detaillierten Ist-Analyse des bestehenden Systems wird zudem die Herangehensweise von anderen Ethikkommissionen an Hochschulen und Universitäten innerhalb Österreichs analysiert, um Vergleichswerte zu sammeln und Unterschiede oder Gemeinsamkeiten definieren zu können.

\medskip

\colorbox{yellow}{TODO - User Centered Design Ansatz, Entwicklung, Fragebogen, etc.}

\section{Motivation}
\label{sec:motivation}

An der Fachhochschule Vorarlberg finden in einer Forschungsgruppe und fünf verschiedenen Forschungszentren, darunter beispielsweise die Forschungszentren \enquote{Nutzerzentrierte Technologien} und \enquote{Business Informatics}, Forschung und Entwicklung mit vermehrtem Augenmerk auf regionaler Zusammenarbeit aber auch internationaler Kooperationen statt. \cite{fachhochschule_vorarlberg_gmbh_forschung_2021} Um etwaige ethische Aspekte von Forschungsprojekten oder Entwicklungsarbeiten abwägen zu können, stehen den genannten Forschungseinrichtungen sowie allen Master-Studierenden der Fachhochschule Vorarlberg (im Rahmen ihres Kontextstudiums oder der Masterarbeit) eine direkt an der Fachhochschule ansässige Forschungsethik-Kommission zur Verfügung. Die Kommission gibt auf Antrag eine Stellungnahme beziehungsweise ein Votum zu einer geplanten wissenschaftlichen Untersuchung oder einem Forschungsvorhaben ab. Das Hauptaugenmerk der Ethikkommission liegt hierbei vor allem auf der Beurteilung von Projekten und Entwicklungsarbeiten, bei denen Menschen beteiligt sind, untersucht werden oder bei denen Folgen für die Beteiligten zu erwarten sind. Der bei einem entsprechenden Forschungsvorhaben einzureichende Antrag wird, zum Zeitpunkt dieser Arbeit im Sommersemester 2023, mittels einer Word-Dokumentenvorlage abgebildet. Um ein Votum von der Forschungsethik-Kommission zu erhalten, muss -- vereinfach zusammengefasst -- die entsprechende Vorlage ausgefüllt und per E-Mail an den Vorsitz der Kommission gesendet werden. \cite{fachhochschule_vorarlberg_gmbh_forschungsethik-kommission_2021}

\medskip

Aufgrund von Feedback von Mitgliedern der Kommission sowie durch Anregungen von ehemaligen Antragssteller:innen ergibt sich für die Forschungsethik-Kommission die Situation, dass der gesamte Prozess (von der Erstellung hin bis zum Einreichen) eines Ethikantrages überdacht werden soll -- vor allem auch in Hinblick auf die technischen und inhaltlichen Limitierungen der Word-Dokumentenvorlage.

Die Motivation dieser Arbeit ergibt sich daher aus dem Umstand, dass das aktuelle System rund um die zu behandelnden Ethikanträge nicht mehr den Anforderungen und Wünschen der Foschungsethik-Kommission der Fachhochschule Vorarlberg entspricht. Als weitere Motivation dient der Ausblick darauf, dass die grundlegende Analyse des Prozesses sowie der Vorschlag möglicher Lösungsansätze beziehungsweise eine etwaige Umsetzung den Grundstein für weitere Schritte in Richtung einer angepassten und nutzerzentrierten Herangehensweise für alle Beteiligten bereitstellen kann. 

\clearpage

\section{Zielsetzung}
\label{sec:zielsetzung}

Die Zielsetzung dieser Masterarbeit lässt sich in zwei verschiedene Kategorien einteilen:
\begin{itemize}
    \item \textit{Unmittelbare Ziele} als direktes Resultat der Masterarbeit
    \item \textit{Langfristige Ergebnisse} aufbauend auf den unmittelbaren Zielen, jedoch außerhalb des Rahmens dieser Masterarbeit
\end{itemize}

\subsection{Unmittelbare Ziele}
\label{sub-sec:unmittelbare-ziele}

Die Masterarbeit hat das grundlegende Ziel, den aktuellen Prozess des Erstellens und Einreichens eines Ethikantrages für die Forschungsethik-Kommission der Fachhochschule Vorarlberg zu analysieren, um die bestehenden Probleme und Schwachstellen detailliert zu identifizieren und festzustellen. Durch die gewonnenen Informationen sowie die Erkenntnisse aus einer ebenfalls detailliert durchgeführten Ist-Analyse anderer Systeme soll zumindest ein Lösungsansatz entwickelt werden, der das bestehende System bei entsprechender Weiterentwicklung potenziell ablösen könnte.

\medskip

Der angestrebte Lösungsansatz, wahrscheinlich in Form eines \ac{poc} oder eines Prototyps mit grundlegenden Funktionen, soll während der Konzeptions- und Implementierungsphase evaluiert werden, um sicherzustellen, dass er die Bedürfnisse der Antragssteller:innen und der Forschungsethik-Kommission zielgerichtet erfüllt. Der \ac{poc} oder Prototyp kann und soll als Grundlage für weitere Entwicklungen und Optimierungen dienen können.

\medskip

Ein weiterer wichtiger Aspekt der Masterarbeit ist die Auseinandersetzung mit den Anforderungen an die Sicherheit und den Datenschutz im Prozess des Erstellens und Einreichens eines Ethikantrages. Dieser Aspekt soll in die Analyse und die Entwicklung von Lösungsansätzen einbezogen werden, um sicherzustellen, dass der neue Prozess nicht nur benutzerfreundlicher, sondern auch auf die Aspekte der Datensicherheit und Datenschutz abgestimmt ist.

\subsection{Langfristige Ergebnisse}
\label{sub-sec:langfristige-ergebnisse}

todo

\section{Frage-/Problemstellung}
\label{sec:frage-problemstellung}

todo

\chapter{Detaillierte Ist-Analyse}
\label{chap:ist-analyse}

todo

% Literaturverzeichnis
\clearpage
\phantomsection
\addcontentsline{toc}{chapter}{Literaturverzeichnis}
\printbibliography

% Eidesstattliche Erklärung
\clearpage
\chapter*{Eidesstattliche Erklärung}
\addcontentsline{toc}{chapter}{Eidesstattliche Erklärung}
Ich erkläre hiermit an Eides statt, dass ich die vorliegende Masterarbeit selbstständig und ohne Benutzung anderer als der angegebenen Hilfsmittel angefertigt habe. Die aus fremden Quellen direkt oder indirekt übernommenen Stellen sind als solche kenntlich gemacht. Die Arbeit wurde bisher weder in gleicher noch in ähnlicher Form einer anderen Prüfungsbehörde vorgelegt und auch noch nicht veröffentlicht.

\vspace{5cm}
\noindent
Dornbirn, am [Tag. Monat Jahr anführen]\hfill Dominic Luidold, BSc

\end{document}
